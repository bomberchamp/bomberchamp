%!TEX root = ../bomberchamp.tex

In standard Q-Learning, the Q-function is updated with a single reward $R_{t+1}$ and the $Q$ value estimate for $s_{t+1}$ \eqref{eq:Q-target}.
This can be extended to multi-step targets\cite{Sutton1988} by accumulating the rewards $R_{t+1},\dots,R_{t+n}$ and the $Q$ value estimate for $s_{t+n+1}$. We get the target
\begin{equation}
Y_{t}^{\text { Multi-Step Q }} \equiv \sum_{k=0}^{n-1} \gamma^k R_{t+k+1}+\gamma^n \max _{a} Q\left(s_{t+n}, a ; \theta\right).
\end{equation}
When combined with the Double DQN target \eqref{eq:DDQN-target}, we get 

\begin{equation}
Y_{t}^{\text { Multi-Step DDQN }} \equiv \sum_{k=0}^{n-1} \gamma^k R_{t+k+1}+\gamma^n Q\left(s_{t+n}, \underset{a}{\operatorname{argmax}} Q\left(s_{t+n}, a ; \theta\right); \theta^{-}\right).
\end{equation}

Q-Learning is thereby similar to multi-step Q-Learning with $n=1$. Higher $n$ lead to faster propagation of reward information to previous time steps which can be useful or even a necessity if the rewards are sparse over a long number of time steps. On the other hand $Q$ may not be able to approach the correct $Q*$ values for an arbitrary policy\cite{Peng1996}. So a fine-tuned $n$ can lead to faster learning.
