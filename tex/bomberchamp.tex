\documentclass[12pt]{article}
%\usepackage[a4paper,left=2cm,right=2cm,top=2cm,bottom=2cm]{geometry}
\usepackage{graphicx}
\usepackage{svg}
\usepackage[utf8]{inputenc}
\usepackage{float}
\usepackage{caption}
\usepackage{amsmath}
\usepackage[backend=biber]{biblatex}
\usepackage{amssymb}
\usepackage{booktabs}
\usepackage{hyperref}
\usepackage{subcaption}
\newsavebox{\largestimage}

\bibliography{literature.bib}
\title{Machine Learning: Final project bomberchamp}
\date{today}
\begin{document}





\section{Introduction}
% Introduction
%!TEX root = ../bomberchamp.tex

For this project we implemented the Rainbow DQN\cite{Hessel2018RainbowCI}, a Deep Q Network that includes a variety of extensions to improve its performance, with the exception of the distributional extension.

% we implemented a rainbow dqn

\subsection{Reinforcement Learning}
In reinforcement learning, sequential decision making problems need to be solved by an agent. The agent is provided with some input state and then needs to choose an action. Input state and chosen action are used to create a subsequent state and a reward signal for the agent. By maximizing the reward signals the agent develops a policy for acting in the environment and thus gets better in solving the given problem. Finding the optimal policy is the goal of reinforcement learning.
Hereby the environment is fully characterized by the state. \\
Time is discrete in reinforcement learning. Therefore the consecutive states form a chain.
Nevertheless transitions from one state to its subsequent state are not dependent on history, they only depend on the current state. This is called the Markov assumption. \\
For finding the optimal policy the state value function $V_\pi(s)$ can be used. It is defined as the expected reward of a policy $\pi$ that can be reached by starting at the state s
\begin{equation}
V_\pi=\mathbb{E}\left[R\right].
\end{equation}
The expected reward $R$ for a given state s is defined as the sum of the current reward and the discounted future rewards
\begin{equation}
R=\sum^\infty_{t=0}\,\gamma^t\,R_t \textrm{ for } s=s_0.
\end{equation}
$\gamma\:\epsilon\left[0,1\right]$ is the discount factor.\\
The optimal policy $\pi^*$ maximizes the state-value function.
% Reinforcement Learning basics a
\subsection{Neural Networks}
%!TEX root = ../bomberchamp.tex

As regression model we use a neural network, consisting of multiple layers of fully connected and convolutional layers. A fully connected layer can be represented as
\begin{equation}
y = w x+b,
\end{equation}
where $w$ and $b$ are learned through backpropagation. By stacking multiple of these layers and separating them with an activation function, e.g. the rectified linear unit
\begin{equation}
\phi_{relu}(y) = \begin{cases}
y & y > 0 \\
0 & otherwise,
\end{cases}
\end{equation}
more complex functions can be approximated.

Instead of fully connected layers, convolutional layers can be used to process images or other spatially connected inputs. Each convolutional layer takes an input of size $w\times h \times c$ and generates an output of size $\frac{w-f+2p}{s}+1 \times \frac{h-f+2p}{s}+1 \times n_c$ by passing $n_c$ filters over regions of $f\times f \times c$ neighboring values in the input. $f$ is the filter size and $n_c$ is the number of filters. The stride $s$ defines the distance that the region is shifted for every step horizontally and vertically. The input is often padded with $p$ rows / columns of zeros on either side to avoid border artifacts. For this project we use \emph{same} padding, so that the output is always of size $\frac{w}{s}\times \frac{h}{s} \times n_c$.

We use a convolutional neural net that is focused on the central region of the input as shown in figure \ref{fig:centre-net}.
\begin{figure}
  \centering
  \includesvg{images/network-arch}
  \caption{Convolutional network focused on the central region.}
  \label{fig:centre-net}
\end{figure}
The input layer gets separated into streams $a$, $b$ and $c$, each taking a centred cropping of the input. Each stream then processes the croppings through a number of convolutional layers depending on the size of the cropping. The results get flattened and concatenated after which they are passed on to the value and advantage streams further described in \ref{ch:duelling}.

% basics j

\section{DQN}
%!TEX root = ./bomberchamp.tex
\subsection{Q-learning}
In Q-learning \cite{DBLP:journals/corr/HasseltGS15} instead of the state-values, Q-values are being used. They are defined as the current reward depending on the chosen action and the discounted future rewards under the premise of always using the policy $\pi$ for further decision making after the initial action choice:
\begin{equation}
Q_\pi(s,a)=\mathbb{E}\left[R\right]|_{a,s,\pi}.
\end{equation} 
The best policy is then determined by maximizing $Q_\pi$ over the actions in every step.
%TODO why is this better than the state value function??
Usually it is not possible to learn every action for every state explicitly because the problems are often to extensive. Therefore the Q-values get approximated by a parametrized function $Q(s,a,\theta_t)$ with $\theta_t$ being the parameters. The Q-function is then updated towards the target value
\begin{equation}\label{eq:Q-target}Y_{t}^{\mathrm{Q}} \equiv R_{t+1}+\gamma \max _{a} Q\left(s_{t+1}, a , \theta_{t}\right)
\end{equation}
after each action via:
\begin{equation}
\theta_{t+1}=\theta_{t}+\alpha\left(Y_{t}^{\mathrm{Q}}-Q\left(s_{t}, a; \theta_{t}\right)\right) \nabla_{\theta_{t}} Q\left(s_{t}, a ; \theta_{t}\right).
\end{equation}
This makes it an off-policy algorithm, as the optimal $Q$-value: $Q^*$ is approximated by $Q$ directly (regardless of the followed policy) in contrast to for example SARSA. The policy is still important as it determines, which state-action pairs are used to update the model \cite{Sutton:1998:IRL:551283}. 
Q-learning is also a model free approach, as it doesn't make a model of the environment but instead directly estimates $Q^*$.\\
%In our case with bomberman, the problem is fully observable, meaning that the agent knows the entire state of the environment at every step of the game.
\subsection{Q-Networks}
For some reinforcement learning problems, the simplest implementation of the Q-learning algorithm, a table is perfectly sufficient to find a good policy.
If we consider more complicated problems with bigger inputs other methods are needed. For state of the art reinforcement learning, usually neural networks are being used.
They are especially convenient as they are trained from raw inputs, which makes handcrafted features redundant \cite{DBLP:journals/corr/MnihKSGAWR13}.
Neural Networks which are implemented to learn with the Q-algorithm are called Q-networks.
They are non linear function approximations for $Q(s,a,\theta_t)$.
To update the Q-network the loss function
\begin{equation}
L_{t}\left(\theta_{t}\right)=\mathbb{E}_{s, a \sim \rho(s,a)}\left[\left(Y_{t}^{\mathrm{Q}}-Q\left(s, a ; \theta_{t}\right)\right)^{2}\right]
\end{equation}
%TODO Erwartungswert großschreiben
is used, where $\rho(s, a)$ is a probability distribution over states and actions. To make this compatible with the Q-learning algorithm, the weights need to be updated at every step and the expectations exchanged with samples from the probability distribution $\rho(s,a)$.
\subsection{DQN}
Deep Q-networks \cite{DBLP:journals/corr/MnihKSGAWR13} are multilayered neural networks which make use of the Q-learning algorithm. The two mayor improvements of the model are the use of two separate networks and an experience buffer.
The two networks are called online network and target network. The target network is used to calculate the targets $Y_{t}^{\mathrm{DQN}}$. It has the same structure as the online network but to make learning more stable, the weights of the target network $\theta_t^-$ stay constant for a longer time. 
They are being copied from the online network every $\tau$ steps.
The target is then calculated by:
\begin{equation}
Y_{t}^{\mathrm{DQN}} \equiv R_{t+1}+\gamma \max _{a} Q\left(s_{t+1}, a ; \theta_{t}^{-}\right).
\end{equation}
If only one network was used, an update of $Q(s,a)$ would often not only lead to a higher value of $Q(s_t,a_t)$ but also to higher expected Q-values $Q(s_{t+1},a)$ for all actions. If the target was also calculated by this network this could then lead to oscillations or divergence of the policy \cite{DBLP:journals/corr/MnihKSGAWR13}.
The additional improvement of DQN is experience replay. Without experience replay, only new experiences are used in training and discarded right afterwards.
Therefore important but rare experiences are almost immediately forgotten and the updates are not independent and identical distributed but strongly correlated. %TODO why is this bad?
To address this problem, an experience buffer is implemented. There the experiences are stored and then at training time sampled uniformly at random. Usually a simple FIFO algorithm is being used. But there are more sophisticated methods for this, one of those is discussed later.
% a

\section{Rainbow / DQN Extensions}
\subsection{Double DQN}
Q-learning  and DQN tend to learn overestimated action values. During the maximization over the action choices, values are rather over than underestimated.
This isn't necessarily a problem if they are uniformly overestimated or if interesting experiences are overestimated. However, in DQN overestimations differ for different actions and states. Combined with bootstrapping, this results in the propagation of wrong values and thus to worse policies \cite{DBLP:journals/corr/HasseltGS15}. To reduce those overestimations, the Double Q-learning algorithm \cite{DBLP:journals/corr/HasseltGS15} is used. 
The main idea of the Double Q-learning algorithm is to decouple value selection and evaluation.
DQN, with separate online and target networks provides an excellent framework for this decoupling:
The online network is used to select an action from the action choices via maximization whereas the target network evaluates the actions to generate the Q-values.
The resulting double DQN yields more accurate values and hence leads to better policies than DQN \cite{DBLP:journals/corr/HasseltGS15}.
Updating double DQNs is similar to updating DQNs when the target is rewritten as:
\begin{equation}\label{eq:DDQN-target}
Y_{t}^{\text { Double DQN }} \equiv R_{t+1}+\gamma Q\left(S_{t+1}, \underset{a}{\operatorname{argmax}} Q\left(S_{t+1}, a ; \theta_{t}\right), \theta_{t}^{-}\right).
\end{equation}
$\theta_{t}^{-}$ and $\theta_{t}$ are the weights of the target and online network respectively.
\subsection{Prioritized Experience Replay}
In standard experience replay, the agent is forced to pick experiences uniformly from all experiences in its memory. Therefore all experiences are sampled with the same frequency that they were originally encountered.
This is not necessarily good for the learning process, as some experiences might not hold any valuable information for the agent but occur very often while other rare situations could be crucial for learning. \\
This can be improved by prioritized experience replay \cite{DBLP:journals/corr/SchaulQAS15}. Here every experience in the buffer gets a priority according to its TD-error.
The TD-error measures the difference between the actual Q-value and the Target-Q-value, so if experiences with bigger TD-errors are provided with bigger priorities, experiences from which there still is a lot to learn are favoured.
The priority $p_i$ is determined from the TD-error $\delta_i$ according to:
\begin{equation}
p_{i}=\left|\delta_{i}\right|+\epsilon.
\end{equation}
Hereby $\epsilon\,>\,0$ denotes a small parameter to ensure, that every experience has a priority bigger than zero and thus can be picked for a sample batch.
New experiences are always added with maximum priority to the memory.
Problematic with this greedy approach is, that only experiences that are picked for learning get their priority updated. Therefore experiences with low initial priority might, because of the buffer memory structure be removed from memory before they could have been picked for learning. The buffer is also very sensitive to noise spikes \cite{DBLP:journals/corr/SchaulQAS15}. \\
To overcome this problems stochastic prioritization is used and $p_i$ adjusted according to:
\begin{equation}
P(i)=\frac{p_{i}^{\alpha}}{\sum_{k} p_{k}^{\alpha}}.
\end{equation}
Here $\alpha\,\epsilon\,[0,1]$ is another parameter which adjusts the amount of prioritization that is used. For $\alpha=0$ we get the uniform case (no prioritization), whereas $\alpha=1$ leads to greedy prioritization. \\
Stochastic prioritization introduces bias to our model. This needs to be considered for updating the model, because it could change the solution the model is converging to \cite{DBLP:journals/corr/SchaulQAS15}. To correct this, importance sampling weights (IS weights) are introduced: 
\begin{equation}
w_{i}=\left(\frac{1}{N} \cdot \frac{1}{P(i)}\right)^{\beta},
\end{equation}
with N being the replay buffer size and $\beta\,\epsilon\,[0,1]$ being a hyperparameter for  adjustment of the bias. 
For $\beta=1$ the bias gets fully compensated. This is most important at the end of the training process. Therefore $\beta$ starts at an initial value and is then being annealed during training. \\\\
An efficient data structure for the memory is crucial for good performance.
To guarantee this, we implemented a sum tree to store the data, where searching is of complexity $O(1)$ and updating of complexity $O(\log N)$.
A sum tree is a binary tree, in which the parent node values are the sum of the child node values. In our sum tree, the transition priorities were saved in the leave nodes. Therefore the root holds the total priority. An array was used to hold the associated data values to the priorities. For the purpose of sampling the total priority is divided into k priority ranges, with k being the number of experiences in one sample.
From each of these priority ranges one value is sampled uniformly and its corresponding leave node is searched. The data belonging to this priority is than used for the sample.

\subsection{Duelling networks}
The idea behind duelling networks \cite{DBLP:journals/corr/WangFL15} is, that for some states only the state-value function is important while for others the chosen action is crucial. \\
Consider a small toy problem: Our agent needs to catch coins which are falling down form above, he can either move right or left or wait to catch them. In some states there are no coins at all. In that state it is not important which action is chosen, whereas for other states its is.
\begin{figure}
\centering
\includegraphics[scale=0.5]{./images/dueling.png}
\caption{Top: Standard Deep Q-network. Bottom: Dueling DQN with two separate streams \cite{DBLP:journals/corr/WangFL15}.}
\label{fig: dueling}
\end{figure}
To exploit this the neural network is split into two streams: 
the action and the state-value stream, as can be seen in figure \ref{fig: dueling}. The network is split after the convolutional layers. Therefore the two streams consist of linear layers. The benefit of this is, that it is possible to get separate estimations for state-value function and action-value function. Hereby the state-value $V(s ; \theta, \beta)$ is a scalar property while the action-vector $A(s, a ; \theta, \alpha)$ has the dimension of the quantity of possible action choices, in our case six. $\theta$ are the weights of the convolutional layers, while $\alpha$ and $\beta$ are the weights for the A- and the V- stream respectively. \\
For getting the Q-value A and V need to be recombined. It is not a good idea though to simply add them together:
\begin{equation}
Q(s, a ; \theta, \alpha, \beta)=V(s ; \theta, \beta)+A(s, a ; \theta, \alpha).
\end{equation}
In that case it would get impossible to retrieve V and A from Q uniquely. One could add a constant to A and subtract it from V without changing Q.
It is better to calculate Q by
\begin{equation}
Q(s, a ; \theta, \alpha, \beta)=V(s ; \theta, \beta)+ \left(A(s, a ; \theta, \alpha)-\frac{1}{|\mathcal{A}|} \sum_{a^{\prime}} A\left(s, a^{\prime} ; \theta, \alpha\right)\right),
\end{equation}
where A and V keep their identifiability and the optimization becomes more stable \cite{DBLP:journals/corr/WangFL15}.
\subsection{Noisy Networks}
So far we did use an $\epsilon$-greedy policy for exploration. Another technique which has been found to produce better result for many of the Atari games are Noisy Nets \cite{DBLP:journals/corr/FortunatoAPMOGM17}. They add parametric noise to the weights and thus aid exploration without the need to pick random actions as part of a policy (e.g. $\epsilon$-greedy). This is very convenient, because there is no need to tune additional hyper parameters as the reinforcement learning algorithm tunes the weights automatically.
Consider a neural network $y=f_{\theta_n}(x)$. Where $\theta_n$ are the noisy weights. A linear layer of a neural network can be written as: 
\begin{equation}
y = w x+b,
\end{equation}
whereas a noisy linear layer can be written as:
\begin{equation}
y =\left(\mu^{w}+\sigma^{w} \odot \varepsilon^{w}\right) x+\mu^{b}+\sigma^{b} \odot \varepsilon^{b}.
\end{equation}
Here $x$ is the input, $w$, and $\mu^{w}+\sigma^{w} \odot \varepsilon^{w}$ are the weights and $b$ and $\mu^{b}+\sigma^{b} \odot \varepsilon^{b}$ are the biases for linear and noisy linear layer respectively. All of the named parameters are trainable except for $\varepsilon^w$ and $\varepsilon^b$ which are noise random variables. We chose factorized gaussian noise for the distribution of the $\varepsilon$ parameters, as it reduces computation time for random number generation, which is important for single thread-agents such as ours \cite{DBLP:journals/corr/FortunatoAPMOGM17}. Here only one independent noise per input and another independent noise per output is needed, in contrast to independent Gaussian noise, where one independent noise per weight would be required.
We factorized $\varepsilon^w$ to $\varepsilon^w_{i,j}$.
The noise random variables can then be written as:
\begin{equation}
\begin{aligned} \varepsilon_{i, j}^{w} &=f\left(\varepsilon_{i}\right) f\left(\varepsilon_{j}\right) \\ \varepsilon_{j}^{b} &=f\left(\varepsilon_{j}\right) \end{aligned},
\end{equation}
where we used \begin{equation}
f(x)=\operatorname{sgn}(x) \sqrt{|x|}.
\end{equation}
The parameters $\mu_{i,j}$ were initialized as samples from a random uniform distribution $\left[-\frac{1}{\sqrt{p}},+\frac{1}{\sqrt{p}}\right]$ with $p$ being the number of inputs for the noisy linear layer. $\sigma_{i, j}$ were set as $\sigma_{i, j}=\frac{\sigma_{0}}{\sqrt{p}}$. \\
For the Noisy Networks implementation we replaced the Fully Dense layers of the state-value and action streams by Noisy layers. %TODO Loss changes?
\cite{DBLP:journals/corr/FortunatoAPMOGM17}
\subsection{Multi-Step Learning} % j
% 
%!TEX root = ../bomberchamp.tex
In reinforcement learning, the reward for a given policy $\pi$ and state $s_{t=0}$ is the discounted return for all future actions. Since $V_\pi(s_t)$ is the expected return for $s_{t}$, we can use the corrected $n$-step truncated return 
\begin{equation}
R_t^{(n)}=\sum^{n-1}_{k=0}\,\gamma^k\,R_{t+k+1} + V_\pi(s_n)
\end{equation}.

Q-Learning takes the reward $R(s_t, a_t)$ from a single step and uses the state value estimate $\hat{V}_\pi(s_{t+1})$ of the next step. If we take the greedy action $a*_{t+1}$ at $s_{t+1}$, we get $\hat{V}_\pi(s_{t+1})=\hat{Q}^{*}(s_{t+1}, a^{*}_{t+1})=\max_a\hat{Q}^{*}(s_{t+1}, a)$

$$\hat{Q}^{*}(s_t, a_t)=R(s_t, a_t)+\gamma \hat{Q}^{*}(s_{t+1}, a_{t+1})$$

For multi-step Q-Learning we take the n-step truncated return
$$R_{t}^{(n)} \equiv \sum_{k=0}^{n-1} \gamma_{t}^{(k)} R_{t+k+1}$$

to estimate the







In Q-Learning we are trying to estimate the expected total discounted return [LINK TO DISCOUNTED RETURN IN INTRO] given a state $s$ and action $a$.

$$Q^{*}(x_t, a_t)=R(x_t, a_t)+\gamma R(x_{t+1}, a_{t+1})+\gamma^2 R(x_{t+2}, a_{t+2}) + ...$$

If we assume that we use the optimal policy for $t+1$ onward, we can substitute $\gamma R(x_{t+1}, a_{t+1})+\gamma^2 R(x_{t+2}, a_{t+2}) + ...$ with $Q^{*}(x_{t+1}, a*_{t+1}$.


\section{Training}
\subsection{Feature space}
%!TEX root = ../bomberchamp.tex


To capture the spatial relations between objects on the field, we use a 2D convolutional neural network. This network takes a matrix $X\in\mathbb{R}^{w\times h \times c} $ as input, where $w$ and $h$ are width and height respectively and $c$ is the number of channels. In the case of bomberman, we have $w=h=17$. For the arena and objects in the game world we give each its own channel (Figure \ref{fig:input-channels}):

\begin{tabular}{l p{0.7\linewidth}}
Walls & $\in\left\{0,1\right\}$ \\
Crates & $\in\left\{0,1\right\}$ \\
Self & $\in\left\{0,1,2\right\}$, this channel gets omitted when we later center the inputs. \\
Others & $\in\left\{0,1,2\right\}$\\
Bombs & $\in\left[0, 1\right]$ \\
Explosions & $\in\left\{0,1\right\}$ \\
Coins& $\in\left\{0,1\right\}$ \\
\end{tabular}

In the channels \emph{self} for the player agent and \emph{others} for the opponents,
$$X_{x,y}|_{c=self/others}=\begin{cases}
2 & \text{player with bomb,} \\
1 & \text{player without bomb,} \\
0 & \text{otherwise.}
\end{cases}
$$

There is no differentiation between the different opponents, since the state only consists of the current time step.
For bombs $(x, y, t)$, $t$ being the time left until the bomb explodes, we calculate $$X_{x, y}|_{c=\text{bombs}} = 1 - \frac{t}{\mathrm{bombtimer}+1} \;\;\forall (x, y, t) \in \text{bombs}$$ with $$X_{x, y}|_{c=\text{bombs}}=0 \;\;\forall (x, y, t\geq0) \notin \text{bombs}.$$
So with $\text{bombtimer}=4$, if $X_{x, y}|_{c=\text{bombs}} = 0.2$, the bomb has just been planted, and if $X_{x, y}|_{c=\text{bombs}} = 1$, the bomb will explode this turn.
While an explosion is active for two turns, this includes when the bomb timer hits zero, so the explosion is only visible to agents during one turn.


\begin{figure}
  \centering
  % Store largest image in a box
  \savebox{\largestimage}{\includegraphics[height=.22\textheight]{images/X_channel-layers.png}}%
  \begin{subfigure}[b]{0.6\textwidth}
    \centering
    \usebox{\largestimage}
    \caption{Visualization of the separate channels}
  \end{subfigure}
  \quad
  \begin{subfigure}[b]{0.36\textwidth}
    \centering
    % Adjust vertical height of smaller image
    \raisebox{\dimexpr.5\ht\largestimage-.5\height}{%
      \includegraphics[height=.18\textheight]{images/X_channel-full.png}}
    \caption{All channels combined}
  \end{subfigure}
  \caption{Input channels (from left to right): walls (gray), crates (brown), self (turquoise), other players (turquoise), bombs (red), explosions (orange), coins (yellow)}
  \label{fig:input-channels}
\end{figure}

The matrix $X$ is sparse in all but two channels, but this feature space captures the spatial correlations well.

A crucial part of the input is the position of the agent itself. If the player position is simply transformed into a channel, it is reduced to one of many variables and it can be difficult to pick up the importance of this specific variable. So to simplify learning for our agent, the agent was centered in the feature space, so that it stays at a fixed position while the environment (coins, crates, other agents) move around it. To implement this, the size of the feature space is increased to $(2w-1, 2h-1, c)$ and the agent placed in the middle of the new board. As the position of the agent is now known implicitely, we can remove the corresponding channel \emph{self}. This also removes knowledge from the state about whether the player has a bomb available, but the knowledge is unimportant for most situations because we later mask the actions.

\subsection{Augmented data}   %a
\vspace{1 cm}
\begin{minipage}{\textwidth}
\includegraphics[scale=0.3]{./images/augmented_original.png}
\includegraphics[scale=0.3]{./images/augmented_lr.png}
\end{minipage}
\begin{minipage}{\textwidth}
\includegraphics[scale=0.3]{./images/augmented_ud.png}
\includegraphics[scale=0.3]{./images/augmented_udlr.png}
\captionof{figure}{Data augmentation: \newline Upper left: original; Upper right: horizontal mirroring; \newline Lower left: vertical mirroring; Lower right: combined mirroring}
\label{fig:augmented}
\end{minipage}
\newline
\newline
\newline
\newline
As the inputs for our bomberchamp agent are symmetric, we wanted to use data augmentation to increase the number of samples for training and to make learning more symmetric.
From each original sample, three augmented samples were created. For augmentation, we had to mirror the environment and to change action choices accordingly. The augmented environment consisted of horizontal mirroring, vertical mirroring and a combination of both as seen in figure \ref{fig:augmented}.
For horizontal mirroring the agent choices left and right were exchanged, for vertical mirroring up and down and for the combination both were swapped. \newline
Data augmentation did not work as well as we expected, so it was not included in our final implementation (more details in the next section).
% inside general seciton about input space \subsection{Centring of agent} % j / a
\subsection{Invalid actions} % 
%!TEX root = ../bomberchamp.tex

We're only selecting from valid actions, since we found that otherwise the invalid actions can cause a lot of trouble. (essentially invalid actions are equal to WAIT, meaning that the rewards for different actions get mixed up)
Another solution would've been to penalize invalid actions and let the agent learn to avoid them, but since filtering out invalid action is really easy and fast, we use that.

\subsection{Auxilliary Reward Design}
%!TEX root = ../bomberchamp.tex

Auxilliary rewards:
For coin collection, 


For single player with coins in crates, interestingly it did learn to destroy the crates and collect the coins pretty well with a reward for destroying crates and no penalty for death. But there were a lot of games where the agent bombed itself in the beginning.
With a penalty for death, the agent learns to avoid death by choosing WAIT. Therefore we penalize this action and the agent learns to destroy the crates and collect coins, but when run too long, it learns to alternate UP-DOWN and RIGHT-LEFT to avoid death.

depending on the crate reward, the agent may ignore coins

In multi-player, without auxilliary rewards, the agent does not learn well.
With auxilliary rewards for crates and WAIT, the agent learns well.

\subsection{Minigames} % j
%!TEX root = ../bomberchamp.tex

\label{ch:minigame}
As even the seemingly simple task of collecting coins in the bomberman arena can be challenging, we made an even simpler minigame. The minigame consists of collecting coins on an otherwise empty arena. Consequently, the feature space is $X\in\left\{0,1\right\}^{w\times h \times 1}$ and we can limit the actions to movement in the directions that do not lead outside the arena. We can also vary $w$, $h$ and the number of coins $n_{\text{coins}}$ to see how a agent is affected by the arena size and coin count.

The game is terminated when all coins have been collected or after a certain duration which is calculated as $$T_{max}=(w+h)*n_{\text{coins}}.$$
This is done to ensure that every coin can be collected, but there is a time limit to collect them.

Having this minigame helped us debug our reinforcement learning agent and find a suitable network for the next task.

\subsection{Network Architecture}
%!TEX root = ../bomberchamp.tex

Having introduced our minigame in \ref{ch:minigame}, we tried different network architectures to see what is necessary for an arena of size $(17, 17)$.
%!TEX root = ../bomberchamp.tex


\begin{figure}
  \centering
  \begin{subfigure}[b]{0.48\linewidth}
    \centering
	\begin{tabular}{l l l}
		\multicolumn{3}{c}{Shared network} \\
		\midrule
		Dense & 64 units & relu \\
		Dense & 64 units & relu \\
		\toprule
		\multicolumn{3}{c}{Advantage / value stream} \\
		\midrule
		NoisyDense & 64 units & relu \\
		\toprule
	\end{tabular}
    \caption{Dense 64}
    \label{fig:dense-arch64}
  \end{subfigure}
  \begin{subfigure}[b]{0.48\linewidth}
    \centering
	\begin{tabular}{l l l}
		\multicolumn{3}{c}{Shared network} \\
		\midrule
		Dense & 256 units & relu \\
		Dense & 256 units & relu \\
		\toprule
		\multicolumn{3}{c}{Advantage / value stream} \\
		\midrule
		NoisyDense & 256 units & relu \\
		\toprule
	\end{tabular}
    \caption{Dense 256}
    \label{fig:dense-arch256}
  \end{subfigure}
  \caption{Simple fully connected (dense) network. The last value and advantage stream layer is omitted.}
  \label{fig:dense-arch}
\end{figure}

%!TEX root = ../bomberchamp.tex

\begin{figure}
  \centering
  \begin{subfigure}[b]{0.48\linewidth}
    \centering
	\begin{tabular}{l l l l l}
		\multicolumn{5}{c}{Shared network} \\
		\midrule
		& $n_c$ & $f$ & $s$ & \\
		\midrule
		Conv2D & 32 & 8 & 4 & relu \\
		Conv2D & 64 & 4 & 2 & relu \\
		Conv2D & 64 & 3 & 1 & relu \\
		\toprule
		\multicolumn{5}{c}{Advantage / value stream} \\
		\midrule
		NoisyDense & \multicolumn{3}{l}{512 units} & relu \\
		\toprule
	\end{tabular}
    \caption{Conv from Rainbow paper\cite{Hessel2018RainbowCI}}
    \label{fig:conv-original}
  \end{subfigure}
  \begin{subfigure}[b]{0.48\linewidth}
    \centering
	\begin{tabular}{l l l l l}
		\multicolumn{5}{c}{Shared network} \\
		\midrule
		& $n_c$ & $f$ & $s$ & \\
		\midrule
		Conv2D & 32 & 1 & 1 & relu \\
		Conv2D & 64 & 4 & 2 & relu \\
		Conv2D & 64 & 3 & 2 & relu \\
		\toprule
		\multicolumn{5}{c}{Advantage / value stream} \\
		\midrule
		NoisyDense & \multicolumn{3}{l}{512 units} & relu \\
		\toprule
	\end{tabular}
    \caption{Modified convolutional NN}
    \label{fig:conv-modified}
  \end{subfigure}
  \caption{Convolutional network with number of channels $n_c$, filter size $f$ and stride $s$. The last value and advantage stream layer is omitted.}
  \label{fig:conv-arch}
\end{figure}

We chose a simple fully connected network with different numbers of neurons (Figure \ref{fig:dense-arch}) for our first tests and expanded on it with the convolutional network that was used on the Atari 2600 benchmark in the Rainbow DQN paper\cite{Hessel2018RainbowCI} (Figure \ref{fig:conv-original}). Since the input for the Atari benchmark was raw pixels with $w=h=84$, this might not be suitable for our problem with meaningful inputs of size $w=h=17$. So we modified the convolutional network to mimick the dimensions of the different layers of the original (Figure \ref{fig:conv-modified}).
%!TEX root = ../../bomberchamp.tex

\begin{figure}
  \centering
  \begin{subfigure}[b]{0.48\linewidth}
    \centering
    	\includegraphics[width=\linewidth]{images/minigame-dense64-arch.png}
    \caption{Dense 64 (\ref{fig:dense-arch64})}
    \label{fig:network-dense64}
  \end{subfigure}
  \quad
  \begin{subfigure}[b]{0.48\linewidth}
    \centering
      \includegraphics[width=\linewidth]{images/minigame-dense256-arch.png}
    \caption{Dense 256 (\ref{fig:dense-arch256})}
    \label{fig:network-dense256}
  \end{subfigure}
  \begin{subfigure}[b]{0.48\linewidth}
    \centering
    	\includegraphics[width=\linewidth]{images/minigame-conv8-4-3-arch.png}
    \caption{Conv from Rainbow paper\cite{Hessel2018RainbowCI} (\ref{fig:conv-original})}
    \label{fig:network-conv843}
  \end{subfigure}
  \quad
  \begin{subfigure}[b]{0.48\linewidth}
    \centering
      \includegraphics[width=\linewidth]{images/minigame-conv1-4-3-arch.png}
    \caption{Conv modified (\ref{fig:conv-modified})}
    \label{fig:network-conv143}
  \end{subfigure}
  \caption{Agents with different network architectures playing the minigame introduced in \ref{ch:minigame} for arena sizes from $4\times4$ to $17\times17$. There are three coins that can be collected per episode. The graphs show coins collected during the last 100 steps, continuing over episodes. For a big arena this number is naturally lower because of the large distances between coins, but it can be seen when and if the agent is plateauing.}
  \label{fig:networks}
\end{figure}



Figure \ref{fig:networks} shows the performance of the agents on different arena sizes of the minigame. While the dense networks (\ref{fig:network-dense64}, \ref{fig:network-dense256}) learn very quickly for small boards, they have difficulties capturing boards of size $w=h\geq10$. When trying the convolutional network designed for Atari games (\ref{fig:network-conv843}), its performance is even worse than the dense networks. This is likely due to the first convolutional layer having a filter size of $8$ and a stride of $4$. For raw pixels sparse with information, a high filter size and stride can help summarize the information and reduce the dimensions. But on our input it may lead to weakening the important values.
So for our modified convolutional net, we set the filter size $f$ and stride $s$ to $1$. The resulting $17\times17$ layer output is close to the originals $21\times21$ layer output, making the remaining network usable. For the minigame, a layer with $f=1$, $s=1$ has no impact because there is only one input channel. But our bomberman input has six channels, so the first layer can transform the information at each location into a more convinient form. The result can be seen in figure \ref{fig:network-conv143}, as it can collect coins efficiently for at least $w=h=10$ after a short while. If trained longer, this network can also solve the minigame for $w=h=17$.

%!TEX root = ../../bomberchamp.tex

\begin{figure}
  \centering
	\begin{tabular}{l l l l | l l l l | l l l l}
		\multicolumn{12}{c}{Shared network} \\
		\midrule
		\multicolumn{4}{c|}{$a$ stream} & \multicolumn{4}{c|}{$b$ stream} & \multicolumn{4}{c}{$c$ stream} \\
		\midrule
		\multicolumn{4}{c|}{Cropping2D  14} & \multicolumn{4}{c|}{Cropping2D  10} & \multicolumn{4}{c}{} \\
		\midrule
		& $n_c$ & $f$ & $s$ & & $n_c$ & $f$ & $s$ & & $n_c$ & $f$ & $s$ \\
		\midrule
		Conv2D & 64 & 3 & 1 & Conv2D & 64 & 3 & 1 &  Conv2D & 32 & 1 & 1 \\
		 &  &  &  &           Conv2D & 64 & 3 & 2 &  Conv2D & 64 & 4 & 2 \\
		 &  &  &  &                  &    &   &   &  Conv2D & 64 & 3 & 2 \\
		 &  &  &  &                  &    &   &   & MaxPool &    &   & 2 \\
		 &  &  &  &                  &    &   &   &  Conv2D & 64 & 3 & 1 \\
		\toprule
	\end{tabular}
	\begin{tabular}{l l l | l l l}
		\multicolumn{3}{c|}{Advantage stream} & \multicolumn{3}{c}{Value stream} \\
		\toprule
		& units & & & units & \\
		\midrule
		NoisyDense & 512 & relu & NoisyDense & 512 & relu \\
		NoisyDense & $D$ & relu & NoisyDense & 1 & relu \\
		\toprule
	\end{tabular}
	\caption{Convolutional net focused on central region as shown in figure \ref{fig:centre-net}. Each convolutional layer is followed by a \emph{relu} activation.}
	\label{fig:conv-focus}
\end{figure}

%!TEX root = ../../bomberchamp.tex

\begin{figure}
  \centering
  \begin{subfigure}[b]{0.48\linewidth}
    \centering
    	\includegraphics[width=\linewidth]{images/fullgame-conv1-4-3-arch.png}
    \caption{Conv modified}
    \label{fig:fullcoins-conv-mod}
  \end{subfigure}
  \quad
  \begin{subfigure}[b]{0.48\linewidth}
    \centering
      \includegraphics[width=\linewidth]{images/fullgame-focus-arch.png}
    \caption{Conv focused}
    \label{fig:fullcoins-conv-focus}
  \end{subfigure}
  \caption{Agents with different network architectures playing the bomberman game in singleplayer coin collection mode with discount factors $\gamma=0.9$ and $\gamma=0.99$.}
  \label{fig:fullcoins}
\end{figure}





\subsection{Self-Play} % j
%!TEX root = ../bomberchamp.tex

We use self-play to train the agent in a four player free-for-all environment. The agent plays against itself, a past version of itself or the heuristic \emph{simple\_agent}. Since the players use the same model, the experiences for the current players can be saved in a shared experience replay buffer. The \emph{simple\_agent} experiences also get saved in the buffer with the $Q$ value estimation being calculated with the current player model. The past versions only serve as opponents with no data added to the buffer.




\section{Observation}
\subsection{It does not work}
% nothing works
% solution: make it better
\subsection{Network scaling} % j
% different networks from minigame to full game
% lots of graphs

\begin{figure}
  \centering
  \includesvg{images/network-arch}
  \caption{Network architecture}
\end{figure}
% our learning process


\section{Summary and Improvements}
% improvements: distributional DQN, train for 3 months
% steps/own bomb in feature space
% RNN (or other things able to capture hidden state), hidden states: time until next bomb of agents, who owns which bomb

% improvement of game setup: j
% - make provided framework easy to use
% -- currently main.py, settings.py and possibly callbacks.py has to be changed for a simple switch between train and test mode
% -- separate rendering and environment, so that the environment can be called from e.g. a jupyter notebook
% -- provide main.py or python notebook 





\section{Model}



\section{Training process}

\subsection{Self play}
%TODO how was it implemented etc
For training purposes a simple agent implementation that follows the rules of bomberman and plays reasonably well was provided.
To exploit this we wanted to first train our agent by classifying inputs generated with the simple agents. %TODO why didn't this work?
This is why we implemented a self play strategy. This was also useful, as we could then use google colab and train more than one agent at the same time while using the same neural network.



\printbibliography

\end{document}
