\documentclass[12pt]{article}
%\usepackage[a4paper,left=2cm,right=2cm,top=2cm,bottom=2cm]{geometry}
\usepackage{graphicx}
\usepackage[utf8]{inputenc}
\usepackage{float}
\usepackage{caption}
\usepackage{amsmath}
\usepackage[backend=biber]{biblatex}
\bibliography{literature.bib}
\title{Machine Learning: Final project bomberchamp}
\date{today}
\begin{document}

\section{Model}
Using neural networks for training of our bomberman agent, we decided to implement Rainbow (without distributional). For this technique, a number of improvements in deep learning is combined to form an integrated agent. Starting with Q-learning, the $Q_\pi$ value is computed by choosing an action a for a given state s and calculating all of the future rewards $R_n$ under a given policy $\pi$.
Hereby future rewards are scaled by a discount factor $\gamma$ to increase the weight of the immediate rewards.

\begin{equation}
Q_\pi(s,a)= E\left[\sum_{n=1}^{\infty} \gamma^{n-1}R_n^n\right] \Bigg|_{a,s,\pi}
\end{equation}

The optimal Q-values are then determined by taking the maximum of all $Q_\pi$-values.
To estimate these, the neural network learns a Q-function $Q(a, s, \theta_t)$ where $\theta_t$ are the weights calculated by the network.
These are updated towards the target $Y_{t}^{\mathrm{Q}}$.
\subsection{DQN}
In Deep Q-learning (DQN) a multi-layered neural network (online network) generates the possible actions for a given state s. Then another network, the target network is used to calculate the targets.
The target network is similar to the online network but its weights $\theta_t^-$ stay constant for a longer time. They are being copied from the online network every $\tau$ steps.
The target is calculated by:
\begin{equation}
Y_{t}^{\mathrm{DQN}} \equiv R_{t+1}+\gamma \max _{a} Q\left(S_{t+1}, a ; \theta_{t}^{-}\right).
\end{equation}
Additionally DQN introduces experience replay. Without experience replay, only new experiences are used in training and discarded right afterwards.
Therefore important but rare experiences are almost immediatly forgotten and the updates are not independent and identical distributed but strongly correlated.
To address this problem, we implemented an experience buffer, in which the experiences are stored and then at training time sampled uniformly at random.
\subsection{Double DQN}
One problem of DQN is, that it overestimates the Q-values because both, action and assosciated Q-value are calculated by the online network.
A simple solution for this is to use the target network to calculate the Q-values for the action selected by the online network. This decoupling reduces the overestimation.
The resulting target is:
\begin{equation}
Y_{t}^{\text { Double DQN }} \equiv R_{t+1}+\gamma Q\left(S_{t+1}, \underset{a}{\operatorname{argmax}} Q\left(S_{t+1}, a ; \theta_{t}\right), \theta_{t}^{-}\right).
\end{equation}
\subsection{Prioritized Experience Replay}
In standard experience replay, the agent is forced to pick experiences uniformly from all experiences in its memory. Therefore all experiences are sampled with the same frequency that they were originally experienced.
This is not necessarily the best for the learning process, as some experiences might not hold any valuable information for the agent but occur very often while other rare situations could be crucial for learning.
This can be improved by prioritized experience replay. Here every experience in the buffer gets a priority according to its TD-error.
The TD-error meassures the difference between the actual Q-value and the Target-Q-value, so if experiences with bigger TD-errors get bigger priorities, we favorize experience from which there still is a lot to learn.
The priority $p_i$ is determined from the TD-error $\delta_i$ according to:
$p_{i}=\left|\delta_{i}\right|+\epsilon $. Hereby $\epsilon$ denotes a small parameter to ensure, that no experience get's priority zero and thus can't be picked for a sample batch.
New experiences are always added with maximum priority to memory.
Problematic with this greedy approach is, that only experiences that are picked for learning get their priority updated. Therefore experiences with low inital priority might, because of the buffer memory structure be removed from memory before they could have been picked for learning. It is also very sensitve to noise spikes.
To overcome this problem stochastic prioritization is used and $p_i$ adjusted according to:
\begin{equation}
P(i)=\frac{p_{i}^{\alpha}}{\sum_{k} p_{k}^{\alpha}}.
\end{equation}
Here $\alpha$ is another parameter which adjusts the amount of prioritization that is used. For $\alpha=0$ we get the uniform case (no prioritization), whereas $\alpha=1$ leads to greedy prioritization.
Stochastic prioritization introduces bias to our model. This needs to be considered for updating it, because it could change the solution the model is converging to. To correct this, importance sampling weights (IS weights) are introduced: 
\begin{equation}
w_{i}=\left(\frac{1}{N} \cdot \frac{1}{P(i)}\right)^{\beta}.
\end{equation}
For $\beta=1$ the bias gets fully compensated. This is most important at the end of the training process. Therefore $\beta$ start at an inital value and is then being annealed during training. \\\\
An efficient data structure for the memory is crucial for good performance.
To guarantee this, we implemented a sum tree to store the data, where searching is of complexity $O(1)$ and updating of complexity $O(\log N)$.
\subsection{Dueling networks}
The idea behind dueling networks is, that for some states it is irrelevant which action to pick.
Consider a small toy problem: Our agent needs to catch coins which are falling down form above, he can either move right or left or wait to catch them. The coins are falling randomly. Sometimes there are no coins falling down at all in one state. In this state it is not important which action is chosen, whereas for other states ist is.
To exploit this the neural network is splitted into two streams: \textbf{TODO: include image} the action and the state-value stream. The benefit of this is, that it is possible to get separate estimations for state-value and action. Hereby the state value $V(s ; \theta_d, \beta)$ is a scalar property while the action-vector $A(s, a ; \theta_d, \alpha)$ has the dimension of the quantity of possible action choices. $\theta_d$ are the weights of the network before splitting into two streams, while $\alpha$ and $\beta$ are the weights for the A- and the V- stream respectively.
Now A and V need to be recombined into a Q-value. It is not a good idea though to simply add them:
\begin{equation}
Q(s, a ; \theta, \alpha, \beta)=V(s ; \theta, \beta)+A(s, a ; \theta, \alpha).
\end{equation}
In that case it gets impossible to retrieve V and A from Q uniquely. One could add a constant to A and substract it from V without changing Q.
It is better to calculate Q by
\begin{equation}
Q(s, a ; \theta, \alpha, \beta)=V(s ; \theta, \beta)+ \left(A(s, a ; \theta, \alpha)-\frac{1}{|\mathcal{A}|} \sum_{a^{\prime}} A\left(s, a^{\prime} ; \theta, \alpha\right)\right),
\end{equation}
where A and V keep their identifiability and the optimization becomes more stable \cite{DBLP:journals/corr/WangFL15}.

\printbibliography

\end{document}
